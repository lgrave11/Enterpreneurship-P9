\section{Paradigm}
In this section, we discuss, choose and outline how to apply one of the two paradigms for software entrepreneurship.

There are two paradigms to choose from as specified in \citet[pg. 30-32]{book:jrose}: Analyse, Design, Enact (ADE) and Consider, Do, Adjust (CDA). 
We choose the Consider, Do, Adjust paradigm, where the model is a modified version called the Build-Measurel-Learn loop from \citet[pg. 75]{ries2011lean}. 
The Build-Measure-Learn feedback loop starts with an idea that inspires building a minimum viable product.
Once a product has been built, the idea is to measure available data and to learn from the experience.
The goal of the model is to minimize the total time through the loop. 

That is entering the Build phase, the first recommender system is the minimum viable product (MVP) introduced in The Lean Startup, \citep[pg. 77]{ries2011lean}.
This MVP version of the product enables a full turn in the Build-Measure-Learn loop with a minimum amount of effort, because this MVP is generally lacking in comparison to the SaaS solution that is intended to be the final product.
In the Measure phase, the company needs to determine if the product being developed has any appeal to the intended customers. 
One way to measure is through a concept called 'Innovation Accounting'\citep[pg. 77]{ries2011lean}.
'Innovation Accounting' is a quantitative approach that allows seeing if the efforts are bearing fruit and allows the company to create learning milistones. 
Finally, entering the Learn phase of the Build-Measure-Learn loop the company must ask itself whether to pivot or persevere with the original strategy.
That is, change the direction of the company or stay the course. 

As such we think, the approach should be as follows:

First of all a single webshop is contacted and offered a custom made recommender solution. 
This is done because developing such a recommender system will help us gain experience and give us ideas on how to proceed on how to create a Software-as-a-Service product that can be generalised to other webshops.
We then enter the Measure phase, we measure the we gain customers as we build custom made recommender systems with the goal of eventually creating a core recommender system that can be used as SaaS.
In the Learn loop, we will decide if we want to pivot or persevere. Eventually, we will have built enough experience to take advantage of the fact that all the recommender systems are similar and can be generalised into a core recommender system that can be used as SaaS.
This will in turn make the maintainability of the recommender systems easier, giving the company the opportunity to provide recommender systems to more companies without losing overview. 

% Outline important financial concerns for your project relative to the choice of paradigm.
Additionally, our choice of paradigm creates financial concerns. 
Given the Build-Measure-Learn loop described, the primary financial concern for the company is to ensure that there is revenue generation during the loops before the core recommender system can be established. 
Specifically, paying wages is a primary financial concern with no steady income from the subscription based license. 
Additionally, there is the concern that it is unknown how long a basic recommender system will take to build to our quality standards, even for a MVP.
It might become a problem if the MVP takes too long for the webshop to be built and they do not want to pay for the time it takes to build.

% Outline phases or activities in the Business Model Design Process.
In \citet[pg. 241-261]{osterwalder2010business}, The Business Model Design Process has 5 different phases: Mobilize, Understand, Design, Implement and Manage.
The Mobilize phase is where the preparation for the design process is done.
What we did for this phase was decide to use a Business Model Canvas based on the initial idea.

The Understand phase is where elements for the design effort are researched and analysed. 
What we did for this phase was identify possible customers, use storytelling to gain clarity about the business idea and how it should function.

The Design phase is where business models are generated, tested and selected.
What we did for this phase was create a business model canvas.

The first three steps alone are the idea part of the Build-Measure-Learn loop, that and what comes after is the Building of the MVP, Measuring the effect, and Learning from it and possibly pivoting or staying the course. 
As such the Implement phase, where the business model is implemented in the field is the Build phase.
The Manage phase, where the business model is adapted or modified in response to the market reaction are then the Measure and Learn parts of the loop.

Obviously the Build-Measure-Learn loop is more conducive to an approach that spends little time in research and instead spends more time on creating a MVP that can be used to launch the business quickly, measuring the effect and learning from it and seeing what should be changed.

%Discuss, choose, and outline how to apply one of the two paradigms for software entrepreneurship
%
%Analyse, Design, Enact (ADE)
%Consider, Do, Adjust (CDA)
%Outline phases or activities in the Business Model Design Process
%Outline important financial concerns for your project relative to the choice of paradigm