\section{Business Idea}
% The idea
The basic business idea for this miniproject is a Software as a Service product for webshops. 
The product itself is a recommender service, where the basic idea is to generate higher revenue for webshops by personalising product recommendations for their customers. 
The value proposition proposed is a core recommender system.
The webshop provide customer-product events to the company and the core recommender system builds a model based on this that can be used to return recommendations for specific customers.
The company then requests recommendations for specific customers, and are provided with model generated recommendations. 
The idea is then to gain access to the product through a subscription based license.
Additionally, consultancy services could be sold to install, maintain, customise and optimise the product. Furthermore, the consultancy services could be sold for other services in general.

% Resource-based Theory
To characterize and discuss this idea, we use Resource-based Theory based on \citet[pg. 13]{book:jrose}.
Resource-based theory is a theoretical concept where the idea is to describe what gives the company its competitive edge by looking at which resources and capabilities are available to it.
Resources included in this analysis of the company must be one of the following: Valuable, rare, costly to imitate, or non-substitutable.

The following resources are available to the company.
\begin{description}[style=nextline]
\item[Software Product]
The company provides a product that is valuable and expensive and difficult to replicate for webshops on their own. 
Additionally, we have a variety of possible algorithms that could be applied to different companies along with the core recommender system that can be adjusted per webshop.
\item[People \& Talent]
The company has software engineers with machine learning backgrounds and this is important for recommender systems. 
Additionally, there are very few machine learners in Denmark and as such these software engineers are rare. 
These software engineers can also be used for general machine learning consultancy jobs.
\item[Core Values]
The company aims to provide a high quality product, because we rely on our reputation to get and retain customers. 
If the product is low quality, customers would go elsewhere or not use it at all. 
\item[Knowledge]
The core competencies of the company lies within the domain of general machine learning applied to recommender systems. 
Additionally, the software engineers working for the company have a experience in business intelligence and general software development processes.
\item[Increased Capabilities]
Though it is intended that the core recommender system can be mostly used for any webshop, it can be adjusted and customised to the individual customer.
\end{description}

These resources lead to the following capabilities.
\begin{description}[style=nextline]
\item[Provide quality recommender systems]
Quality recommender systems have not yet been utilised to a great extent in Danish webshops.
\item[Improvement of knowledge base]
The field of recommender systems is still fairly young, and it is being improved all the time. 
The company has the capability of improving general knowledge in regards to machine learning, recommender systems and business intelligence. 
Thereby the company can provide better value over time for the customers by iterative improvement of the value they receive through the recommender system by updating the core.
\item[Response to Customer Needs]
The company also has the capability of adjusting the model for a given customer, to improve their experience. This is done through a consultancy service.
\end{description}

These are overall valuable, rare, costly to imitate and non-substitutable which gives the company a competitive edge.

An alternative model to examine the business idea could have been the Five Forces model, where the threats to the company and its market are examined.
However, it was felt that there was no obvious threat at this point, as such the resource based theory seemed to fit better\citep[pg. 16]{book:jrose}.


%Describe your idea. Characterize and discuss it using at least one of the following models or theories (e.g. as part of sections on understanding, designing, or managing):
%Porter’s Value Chain
%Resource-based Theory
%Knowledge-based Theory
%Five Forces
%Change and Resistance Theory